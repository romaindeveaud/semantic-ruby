\documentclass[xcolor=dvipsnames]{beamer}

\usecolortheme[RGB={107,142,35}]{structure}
\usepackage{beamerthemesplit}
\usepackage{graphicx}
\usepackage[utf8]{inputenc} 

\title{Interrogations en langue naturelle}
\subtitle{Projet M1}
\author{Ludovic Bonnefoy \and Romain Deveaud}
\date{Jeudi 18 juin 2009}
\institute{Tutoré par Marc El-Bèze et encadré par Eric Charton}

\begin{document}

\frame{\titlepage}

\section[Sommaire]{}
\frame{\tableofcontents}

\section{Introduction}
\frame
{
    \frametitle{Introduction}
    \begin{itemize}
      \item<1-> Trululu
    \end{itemize}
}
\section{La recherche d'information, le langage naturel et NLGbAse}
\subsection{Moteurs de recherche intégrant la sémantique}
\frame
{
    \frametitle{Moteurs de recherche intégrant la sémantique}
    \begin{itemize}
      \item<1-> Google, Powerset, Hakia...
      \item<2-> Algorithmes ayant recours à des sources extérieures.
      \item<3-> Enrichissement et activité communautaire indispensables pour la validité et la récence des informations.
      \item<4-> NLGbAse : base de données classifiée (ontologie) issue de Wikipédia.
    \end{itemize}
}
\subsection{Présentation de NLGbAse}
\frame
{
    \frametitle{Présentation de NLGbAse}
    \begin{itemize}
      \item<1-> Trois outils de recherche d'informations.
      \item<2-> Un moteur "classique", prennant en entrée des mots-clés et appliquant un algorithme de compacité.
      \item<3-> Un moteur "sémantique", reprennant le même algorithme que le précédent, mais permettant de sélectionner les résultats appartenant à une catégorie sémantique précise.
      \item<4-> Un moteur "extracteur d'informations", basé sur un algorithme de compacité, permettant d'obtenir une information précise éventuellement contenu dans un document.
    \end{itemize}
}
\section{Algorithmes déployés}
\subsection{Catégorisation d'une question}
\subsection{Extraction de mots-clés}
\section{Conclusion}
\frame
{
    \frametitle{Conclusion}
    \begin{itemize}
      \item<1-> Trululu
    \end{itemize}
}
\end{document}
