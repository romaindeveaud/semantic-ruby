\documentclass[xcolor=dvipsnames]{beamer}

\usecolortheme[RGB={107,142,35}]{structure}
\usepackage{beamerthemesplit}
\usepackage{graphicx}
\usepackage[utf8]{inputenc} 

\title{Interrogations en langue naturelle}
\subtitle{Projet M1}
\author{Ludovic Bonnefoy \and Romain Deveaud}
\date{Jeudi 18 juin 2009}
\institute{Tutoré par Marc El-Bèze et encadré par Eric Charton}

\begin{document}

\frame{\titlepage}

\section[Sommaire]{}
\frame{\tableofcontents}

\section{Introduction}
\frame
{
    \frametitle{Introduction}
    \begin{itemize}
      \item<1-> Trululu
    \end{itemize}
}
\section{La recherche d'information, le langage naturel et NLGbAse}
\subsection{Moteurs de recherche intégrant la sémantique}
\frame
{
    \frametitle{Moteurs de recherche intégrant la sémantique}
    \begin{itemize}
      \item<1-> Google, Powerset, Hakia...
      \item<2-> Algorithmes ayant recours à des sources extérieures.
      \item<3-> Enrichissement et activité communautaire indispensables pour la validité et la récence des informations.
      \item<4-> NLGbAse : base de données classifiée (ontologie) issue de Wikipédia.
    \end{itemize}
}
\subsection{Présentation de NLGbAse}
\frame
{
    \frametitle{Présentation de NLGbAse}
    \begin{itemize}
      \item<1-> Trois outils de recherche d'informations.
      \item<2-> Un moteur "classique", prennant en entrée des mots-clés utilisant la similarité cosine.
      \item<3-> Un moteur "sémantique", reprennant le même algorithme que le précédent, mais permettant de sélectionner les résultats appartenant à une catégorie sémantique précise.
      \item<4-> Un moteur "extracteur d'informations", basé sur un algorithme de compacité, permettant d'obtenir une information précise éventuellement contenu dans un document.
    \end{itemize}
}
\section{Algorithmes déployés}
\subsection{Catégorisation d'une question}
\frame
{
    \frametitle{Les règles}
    \begin{itemize}
        \item Application de règles sur les pronoms interrogatifs.
        \begin{itemize}
            \item "Who","Whom","Whose" => pers
            \item "How long","How much", "How many" => amount
            \item "What","Why",... => ?
        \end{itemize}
    \end{itemize}
}
\frame 
{
    \frametitle{Catégorisation des noms propres}
    \begin{itemize}
        \item Si insuffisant : extraction du nom propre de la question.
        \begin{itemize}
            \item on le catégorise avec NLGbAse
            \item si échec on vérifie l'orthographe sur google.com
            \item si modification on interroge de nouveau NLGbAse
            \item enfin si NLGbAse n'a rien retourné on interroge CCG
        \end{itemize}
    \end{itemize}
}
\frame 
{
    \frametitle{Catégorisation de l'objet de la question via Wordnet}
    \begin{itemize}
        \item Si la phrase ne contient pas de noms propres ou que l'étape précédente n'a rien donné : 
        \item On récupère l'objet de la question et on va essayer de le catégoriser en ayant recours à Wordnet
        \item Wordnet associe une catégorie à la majorité des mots, cependant elles ne correspondent pas à Ester
        \item Nous utilisons donc une liste de correspondances : mot -> catégorie
        \begin{itemize}
            \item Mots de plus au niveau dans les arbres d'hyperonymes qui n'ont pour hyponymes que des mots de même classe.
        \end{itemize}
        \item L'algorithme est le suivant :
    \end{itemize}
}
\frame
{
    \frametitle{Catégorisation de l'objet de la question via Wordnet(2)}
    \begin{itemize}
        \item Le mot est-il dans la liste?
        \item Si oui Fin
        \item Sinon on réessaye avec son hypéronyme.
        \item Tant qu'un hypéronyme n'est pas dans la liste ou que l'on a pas atteint le concept de plus haut niveau on réitère.
        \item Finalement si aucune classe n'est trouvée, on prend celle que Wordnet propose.
        \item Si toutes les stratégies ont échoués on prend la classe unk.
    \end{itemize}
}

\subsection{Extraction de mots-clés}

\section{Améliorations envisageables}
\frame
{
    \frametitle{Améliorations}
    \begin{itemize}
        \item Accepter plusieurs classes
        \item Ajouter des options de ri classique
        \begin{itemize}
            \item Permettre d'élargir la requete (mots-clés et catégories)
            \item Opérateurs logiques
        \end{itemize}
    \end{itemize}
}
\section{Conclusion}
\frame
{
    \frametitle{Conclusion}
    \begin{itemize}
      \item<1-> Trululu
    \end{itemize}
}
\end{document}
